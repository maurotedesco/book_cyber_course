\chapter{Presentazione Corso}

%\ = {\tenrm My First Reader\hfil Page \folio}

\section{DURATA}
 5 giorni
\section{Introduzione}

Sempre più la sicurezza informatica sta diventando per le imprese un asset imprescindibile, alcune aziende se ne devono necessariamente
dotare per la protezione e salvaguardia dei propri dati.

Le aziende che offrono servizi di sicurezza devono stare ai passi con i tempi e questo non è facile.

All’interno di questo scenario si inserisce il corso che fornirà conoscenze e competenze nell’ambito
della sicurezza informatica con un focus sulle tecniche di hacking utili ad implementare delle policy
di sicurezza efficaci ed efficienti.

La prima parte del corso ripercorre tutto quello che è relativo alla connessione alle reti IP.

Le connessioni Internet ed Intranet, oltre a permettere l'accesso ad una mole enorme di informazioni
rende i nostri dispositivi potenzialmente attaccabili.

Le vulnerabilità dei sistemi informatici consentono di accedere a segreti industriali, brevetti e
innovazioni che hanno richiesto anni di ricerca.

Il crimine informatico può decretare il fallimento di aziende e quindi danni economici enormi.

\section{Target}
Il corso è rivolto al personale tecnico già in possesso di competenze sulle reti IP e che in azienda si
occupa dei sistemi e della sicurezza informatica.

\section{Obiettivi}
Fornire conoscenze e competenze nell’ambito della sicurezza informatica con un focus sulle
tecniche di hacking utili ad implementare delle policy di sicurezza.
Al termine del corso il partecipante avrà acquisito conoscenze e strumenti che gli permetteranno di
pensare ed agire in maniera autonoma in ambito sicurezza informatica ottimizzando l'analisi di una rete aziendale.

\section{I partecipanti saranno in grado di:}

\begin{itemize}
\item Comprendere i principi di sicurezza della rete.
\item Mettere in sicurezza i servizi di una rete.
\item Conoscere le caratteristiche dei firewall e ids/ips.
\item Effettuare scansioni.
\item Effettuare dei test di sicurezza, VA e PT.
\item Utilizzare strumenti di auditing e riconoscere tentativi di intrusione.
\item Comprendere i meccanismi di accesso sicuro tramite vpn.
\end{itemize}

\section{Struttura e contenuto}

Il corso si articola nei seguenti moduli:
\begin{itemize}
	
\item Reti e IP

\item Minacce alla sicurezza nelle infrastrutture di rete

\item Tipi di attacchi

\item Tecniche di mitigazione

\item Meccanismi di authentication, authorization, accounting: AAA

\item Tecniche di firewalling \index{firewall}

\item Meccanismi di intrusion detection (ids) \index{ids} e prevention (ips) \index{ips}

\item Metodi per garantire riservatezza e integrità dei dati

\item Raccolta di informazioni e scanning

\item Sfruttare la cattiva amministrazione del sistema

\item Rilevazione delle vulnerabilità \index{vulnerabilità}

\item Sfruttamento delle vulnerabilità per accedere ai sistemi

\item Servizi di cifratura \index{cifratura} simmetrica e asimmetrica

\item Digital signature, meccanismi di hash e autenticazione

\item Introduzione alle virtual private networks.

\end{itemize}
	
\section{Metodologia}

Durante il corso i partecipanti verranno coinvolti in attività di laboratorio e case study per rafforzare
le competenze apprese durante le lezioni teoriche.
