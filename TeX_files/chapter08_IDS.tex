\chapter{Intrusion detection (ids) e prevention (ips)}  %capitolo 8

Un attacco deve essere rilevato con degli strumenti che segnalano degli eventi, valutato attraverso un'analisi degli eventi segnalati e scaturire un incidente di sicurezza dove vengono riportate le diverse anomalie riscontrate.
Tutto questo deve essere di aiuto agli esperti che possano prendere delle decisioni su come comportarsi e quali azioni intraprendere.

\subsection{Attacchi}
Interni
esterni

\subsection{Eventi}
Piattaforme di Rilevamento
(NIDS, HIDS, Correlatori, Anti-Virus, Firewall, Log,)

\subsection{Incidenti}

Anomalie di rete
Anomalie di Sistema
Anomalie di Infrastruttura
Notifiche Dirette
Indicatori Fisici
Indicatori di Business

Un IDS non si sostituisce ai normali controlli, ma piuttosto cerca di scoprire i loro fallimenti
Chi entra in un sistema informatico abusivamente compie alcuni tipi di azione che un utente normale non farebbe mai; identificando queste azioni “anomale” possiamo scoprire un intruso
Due metodi principali per farlo:
Anomaly Detection: determinare statisticamente modelli di “comportamento normale”, e segnalare eventuali deviazioni “significative”; conoscenza a posteriori
Misuse Detection: confrontare gli eventi con “schemi” predefiniti di attacchi; conoscenza a priori

I Firewall loggano traffico basandosi su Source/Destination IP/Port e trattano ogni pacchetto come evento individuale (escludendo la tecnologia Statefull inspection)
Risultato: molti dati, molte segnalazioni irrilevanti per stimare le intrusioni in corso
Gli IDS esaminano il payload applicativo dei pacchetti effettuando una serie di analisi per determinare la presenza di traffico anomalo
Risultato: meno eventi = meno falsi positivi = maggiore efficacia

\subsection{Posizionamento}

Esternamente al firewall
Individuano gli attacchi diretti verso la propria rete
Effettivamente rilevano un maggior numero di eventi
Generano una maggior quantità di logs

Internamente rispetto al firewall
Controllano solo ciò che il firewall lascia passare (verso l’interno)

\subsection{Network based Vs. Host Based}

HOST based: i primi IDS erano host-based; un IDS host-based si appoggia al sistema operativo e controlla le system call (esecuzione e controllo dei processi) e gli accessi (al sistema, ai device…)
NETWORK based: controllano il traffico sulla rete cercando nel flusso di pacchetti le tracce di una intrusione
Prossima frontiera: interoperabilità, correlazione, normalizzazione, AI/ML
Entrambi possono essere Anomaly o Misuse based.

NETWORK
Caratteristiche
Gli NIDS esaminano pacchetti raw nella rete in modo passivo e generano eventi

Vantaggi
Facilmente dispiegabile
Non intrusivo
Estremamente difficile da evadere

Svantaggi
Non riconosce traffico criptato
Non riconosce il traffico intra-lan

HOST
Caratteristiche
Funziona su un singolo host
Puo’ analizzare audit, log, integrità di file e directory ecc.
Vantaggi
Funzionamento più accurato di un NIDS
Minor volume di traffico e quindi minor mole di lavoro
Svantaggi
Dispiegamento costoso
Cosa accade se l’host è compromesso?



\subsection{Signature}

Caratteristiche
Confronta con pattern già noti
per verificare l’attacco
Vantaggi
Molto diffuso
Molto veloce
Facile da implementare
Svantaggi
Non puo’ intercettare attacchi per I quali non sono disponibili firme d’attacco

\subsection{Anomaly}

Caratteristiche
 Usa modelli statistichi o motori di apprendimento per individuare comportamenti legittimi
 Riconosce le intrusioni a partire da quelle normali a finire a quelle potenziali
  Vantaggi
 Puo’ individuare tentativi di exploit nuovi e vulnerabilità non ancora note.
 Può riconoscere usi autorizzati che cascano al di fuori della norma
  Svantaggi
 Generalmente più lento, più oneroso in termini di risosrse paragonato a un signature-based IDS
Maggiore complessità e di difficile configurazione
Maggiore percentuale di falsi allarmi

