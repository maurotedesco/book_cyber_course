\chapter{Introduzione alle virtual private networks}  % capitolo 16

\section{Openvpn 2.X}
OpenVPN è un software "full-featured SSL VPN" che implementa una rete sicura a layer OSI 2 o 3 usando il protocollo standard SSL/TLS, supporta diversi metodi di autenticazione ed è molto flessibile utilizzando certificati, smart cards, e/o credenziali con username/password, inoltre permette di abilitare policy per utenti o gruppi-specifici scrivendo specifiche regole di firewall sulla interfaccia della VPN. OpenVPN non è una web application o proxy e non lavora attraverso web browser. 

In questa documentazione viene utilizzata la versione 2.X di OpenVPN, che espande le capacità di OpenVPN 1.X offrendo la scalabilità client/server e permettendo a più client di collegarsi allo stesso server OpenVPN.

\section{Obbiettivo}
Un utente dotato di una credenziale di accesso (certificato e software VPN) fornita dagli operatori che gestiscono il concentratore VPN si può collegare ai servizi interni della rete LAN. Una volta autentificato il sistema registrerà la sua connessione e si potrà monitorare la connessione attraverso l'indirizzo IP assegnato. Per un maggiore grado di controllo il traffico effettuato dall'utente verrà loggato.

Il client collegato potrà simulare applicativi ed accedere a servizi come se fosse sulla rete locale.

\section{Installazione software}
L'installazione del software può essere fatta in diversi modi, utilizzando i gestori di pacchetti per la distribuzione scelta, installando i pacchetti a mano o utilizzando il codice sorgente.
Nel nostro caso abbiamo utilizzato il gestore di pacchetti "aptitude" specifico per distribuzione Debian, l'interfaccia che compare permette di scegliere il pacchetto da installare e di risolvere le dipendenze.

L'installazione tramite riga di comando deve tenere conto della dipendenza dei pacchetti e nello specifico il pacchetto dipende da:

\textasteriskcentered openssl
\textasteriskcentered lzo
\textasteriskcentered pam

Installazione su Ubuntu/debian

\begin{itemize}
    \item Installazione Easy-Rsa \\
      \textbf{ \#nmap  apt-get install easy-rsa}
   \item Installazione Openvpn \\
      \textbf{ \#nmap  apt-get install openvpn}
\end{itemize}


Riportiamo di seguito anche un esempio di installazione tramite riga di comando per una distribuzione che usa il comando rpm(RedHat, Fedora, Suse):

\begin{itemize}
    \item Installazione Openvpn \\
      \textbf{ \#rpm -ivh openvpn-[details].rpm \#installazione del pacchetto}
   \item Upgrade Openvpn \\
      \textbf{ \#rpm -Uvh openvpn-[details].rpm  \#Upgrade del pacchetto}
\end{itemize}

\section{Certificate Authority (CA)}

Il primo passo da fare è la creazione di una PKI (public key infrastructure).
 Una PKI consiste di:
\begin{itemize}
    
   \item  un certificato, una chiave,  per la Certificate Authority (CA) che viene usato per segnare ogni certificato per il server e per i client. \\
      \textbf{ \#genara certificato CA  \#Generazione del certificato della CA}
    \item  un certificato separato (conosciuto come public key) e una chiave privata per il server \\
      \textbf{ \#rpm -ivh openvpn-[details].rpm \#Generazione del certificato del server}
      
    \item  un certificato separato (conosciuto come public key) e una chiave privata per ogni client \\
      \textbf{ \#rpm -ivh openvpn-[details].rpm \#Generazione del certificato del client} 
\end{itemize}

