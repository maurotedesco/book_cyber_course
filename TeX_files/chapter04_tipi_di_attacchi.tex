\chapter{Tipi di attacchi}	 %capitolo 4

I tpi di attacchipiù comuni sono:\\

% 7 Layer ISO/OSI
\begin{tabular}{|c|c|c|c|}
	\hline 
	7& applicazioni &  &  \\ 
	\hline 
	6& Session &  &  \\ 
	\hline 
	5& Presentation &  &  \\ 
	\hline 
	4& IP  &  &  \\ 
	\hline 
	3& TCP &  &  \\ 
	\hline 
	2&  Mac&  &  \\ 
	\hline 
	1& Fisico &  &  \\ 
	\hline 
\end{tabular} \\

4 Layer per il TCP/IP \\

% 4 Layer TCP/IP
\begin{tabular}{|c|c|c|c|}
	\hline 
	4& Applicazione &  &  \\ 
	\hline 
	3& IP &  &  \\ 
	\hline 
	2&  TCP&  &  \\ 
	\hline 
	1& Fisico &  &  \\ 
	\hline 
\end{tabular} 
\subsection{IP Spoofing}

Ogni postazione puo’ inviare pacchetti spacciandosi per un indirizzo IP qualunque.
 I Reply saranno ruotati alla subnet appropriata
 Routing asimmetrico
 Gli attaccanti potrebbero cosi’ non ottenere risposta se la vittima risiede su una subnet diversa.
Per alcuni attacchi questo non e’ importante
 Analogie
 Nessuno vi impedisce di inviare lettere con indirizzo mittente diverso dal vostro.
 
 \subsection{Difendersi da "IP Spoofing"}
 
 Filtrare I pacchetti in ingresso
 Impedire l’acesso a pacchetti provenienti con indirizzo di broadcast in destinazione verso la propria rete
 Impedire l’accesso ai pacchetti provenienti da reti non ruotabili o private.
Egress filtering
 Impedire agli host della propria rete di spoofare il proprio indirizzo IP per I pacchetti verso le altre reti
Impedire i broadcast verso l’esterno