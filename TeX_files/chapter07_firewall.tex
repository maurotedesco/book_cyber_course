\chapter{Firewall e tecniche di firewalling}  %capitolo 7

\section{Che cosa è un firewall? }
Partiamo dal significato della parola introdotta in informatica come similitudine di quel muro che separa le case di legno per evitare il propagarsi di un incendio.
Il firewall è un muro di mattoni che separa due costruzioni isolandole in caso di incendio, un "muro tagliafuoco". In informatica il firewall isola una rete da un'altra; la rete interna di una organizzazione da Internet o da reti esterne, consentendo il transito solo a determinati servizi.

\section{Tipologie di firewall}

\begin{itemize}
    \item packet filter
    \item application gateways
    \item statefull inspection
\end{itemize}

\subsection{Packet Filter}

La rete interna è collegata a Internet con un router.
Alcune case produttrici di router prevedono moduli opzionali per filtrare i pacchetti sulla base di:

Indirizzo IP sorgente
Indirizzo IP destinazione
Porte TCP/UDP (sorgente/destinazione)
ICMP message type
TCP SYN e ACK bits

\subsection{ Statefull filter}

Il pacchetto viene prelevato fra il livello 2 della scheda di rete ed il livello di rete
L’intero pacchetto viene analizzato e sono memorizzate tutte le informazioni utili per i controlli successivi
Nell’applicare l’analisi ai pacchetti successivi si utilizzano le informazioni memorizzate.
 

\section{Utilizzo}
\begin{itemize}
    \item Per prevenire attacchi DoS
    \item Per prevenire modifiche illegali a dati interni.
    \item Per impedire agli aggressori di raccogliere informazioni riservate
\end{itemize}




\section{Regole di Accesso}
Usare il logging
DROP di tutto tranne quello che server
Regole puntuali
Poche regole 
\section{ Firewall per ridurre informazioni}
Ignorare
L'ICMP
richieste in Broadcasting
DROP porte chiuse
Modifica banner dei servizi




 
 \section{Limitazione Firewall}
 
 IP spoofing: I router non possono sapere se i dati provengono realmente dalla sorgente dichiarata

Diverse applicazioni hanno bisogno ciascuna del poprio gatweway applicativo.

I software client devono conoscere il metodo per accedere al gateway applicativo.
Es, deve essere impostato l’indirizzo del proxy all’interno del browser

\begin{itemize}
    \item Il traffico malizioso che transita su porte aperte non viene ispezionato dai firewall non applicativi

     \item Tutto il traffico che transita attraverso un tunnel criptato non può essere analizzato

    \item Dopo che una rete è stata penetrata il traffico anomalo appare come legittimo.

     \item Non fornisce protezione da utenti o amministratori che accidentalmente installano virus

   \item Non fornisce protezione se vengono utilizzate password amministrative deboli oppure vengono utilizzati protocolli che transitano in chiaro
   
   
\end{itemize}


